\documentclass{scrartcl}
\usepackage[a4paper, margin = 3cm]{geometry}
\usepackage[english]{babel}
\usepackage{amsmath}
\usepackage{amssymb}
\usepackage{hyperref}
\usepackage{graphicx}

\setkomafont{disposition}{\normalfont\bfseries}
\newcommand{\tcooe}{\ensuremath{\mathrm{tCO_2e}}}

\begin{document}

\title{Emissions Reduction in the Global Steel Industry and Policy Measures}
\subtitle{A group project topic}
\author{Eunseong Park\footnote{eunseong.park@zew.de}}
\date{December 2025}
\maketitle

\section*{Context}
The production of steel is one of the most carbon-intensive industries, 
accounting for approximately 7\% of global greenhouse gas emissions. 
Meeting the Paris Agreement goals to limit global warming to below 2°C 
will require significant decarbonization of the steel industry. 
However, there is a concern that unilateral climate policies, such as 
(sub-global) carbon taxes or emissions trading systems, 
may lead to carbon leakage, whereby emissions simply shift to other 
countries with less stringent climate policies. 
This has led to calls for border carbon adjustments, 
which seek to level the playing field for domestic producers and ensure 
that carbon leakage does not occur. 
In this context, policymakers around the world are exploring different 
climate policy instruments to reduce emissions in the steel industry.

\section*{Starting Point}
We will provide a simple single-country partial equilibrium model of the steel industry. 
\subsection*{Tasks}
\begin{itemize}
    \item Extend the model to a multi-region model. 
    Simulate the impact of various policy instruments on the steel industry's 
    greenhouse gas emissions, such as carbon taxes, emissions trading systems, 
    and technology standards.
    \item Assess the effectiveness of climate policy instruments for reducing
     emissions in the steel industry. Calculate the potential carbon leakage 
     rate and discuss the implications of this phenomenon for climate policy design.
    \item Evaluate the effectiveness of border carbon adjustments in mitigating 
    this risk. Analyze the trade-offs between protecting domestic producers and 
    ensuring global emissions reductions.
\end{itemize}

\pagebreak

\section*{A Simple Steel Industry Model}
This simple model is largely based on the model by Mathiesen and Maestad (2004).\footnote{
    Mathiesen, Lars, and Ottar Maestad. “Climate Policy and the Steel Industry: 
    Achieving Global Emission Reductions by an Incomplete Climate Agreement.” 
    The Energy Journal 25, no. 4 (2004): 91–114. 
    \url{https://doi.org/10.5547/ISSN0195-6574-EJ-Vol25-No4-5}.}
\subsection*{Sets}
\begin{itemize}
    \item Steel plants \(i \in I = \{1, 2, 3, ..., n\}\)
    \item Factors \(f \in F = \{\text{Iron ore, Coking coal, Steel scrap, Electricity, Natural gas}\}\)
    \item Technologies \(t \in T = \{\text{BOF, EAF, DRI}\}\)
\end{itemize}
Each steel plant will be mapped to a technology with the following notation: 
the technology of the steel plant \(i\) is \( t(i) \).

\subsection*{Parameters}
\begin{itemize}
    \item Factor use per ton crude steel production \(a_{t,f}\):
    \begin{center}
    \begin{tabular}{ cccccc }
        \hline
        Technology & Iron ore & Coking Coal & Steel scrap & Electricity & Natural Gas \\
                   & (ton)    & (ton)       & (ton)       & (MWh)       & (ton)       \\
        \hline
        BOF        & 1.401    & 0.653       & 0.252       & 0.033       &             \\
        EAF        &          &             & 1.026       & 0.523       &             \\
        DRI        & 0.852    &             & 0.579       & 0.707       & 7.199       \\
        \hline
       \end{tabular}
    \end{center}
    \item Reference factor market prices (USD per ton or MWh) \(\bar{V}_f\):
    \begin{center}
        \begin{tabular}{ cccccc } 
            \hline
            Iron ore & Coking Coal & Steel scrap & Electricity & Natural Gas \\
            33.0     & 63.0        & 136.0       & 70.0        & 4.3         \\
            \hline
           \end{tabular}
    \end{center}
    \item Carbon emissions coefficients (\tcooe{} per ton or MWh) \(\kappa_f \): 
    \begin{center}
        \begin{tabular}{ cccccc } 
            \hline
            Iron ore & Coking Coal & Steel scrap & Electricity & Natural Gas \\
            0.02     & 2.76        & 0.01        & 0.29        & 2.34        \\
            \hline
           \end{tabular}
    \end{center}
    \item Price elasticity of supply for each factor \(\rho_f\):
    \begin{center}
        \begin{tabular}{ cccccc } 
            \hline
            Iron ore & Coking Coal & Steel scrap & Electricity & Natural Gas \\
            1.0      & 2.0         & 0.5         & 0           & 0           \\
            \hline
           \end{tabular}
    \end{center}
    \item Price elasticity of demand for steel \(\varepsilon = -0.3\)
    \item Plant-specific additional costs of each factor (e.g. transport) \(\tau_{i,f}\)
    \item Plant-specific production capacity \(\hat{Y}_i\)
\end{itemize}

\pagebreak

\subsection*{Calibration and reference point}
To start the model, we need to set initial values for the variables. 
Doing so will allow us to solve our model as expected. 
However, since we are simplifying reality with many assumptions, 
the model will not completely replicate every detail of empirical data. 
Therefore, to conduct our economic analysis, 
we will construct a price for crude steel and production quantities of each plants 
that are consistent with the cost structure and demand constraint of the model,
and use them as the reference point.

In the gams file, some fictive plants are created. To make these plants heterogenous, 
we assume that they have some additional additive costs 
\(\tau_{i,f} \sim U(0, 0.1)\) when using each factor. 
They are also different with respect to their capacities 
\(\hat{Y}_i \sim U(10, 15)\). 
Then we assume for the moment that their production quantities are
\(\bar{Y}_i = \hat{Y}_i - \delta_i\) where \(\delta_i \sim U(0, 5)\).
Our reference demand \(\bar{D}\) is then the sum of all production quantities: 
\( \bar{D} = \sum_{i \in I} \bar{Y}_i\) and we solve the following problem:
\begin{align*}
    \min_{\{Y_i\}_{i\in I}} & \sum_{i \in I} \sum_{f \in F} a_{t(i), f} (\bar{V}_f + \tau_{i,f}) Y_i \\
    \text{subject to } \quad  & \sum_{i \in I} Y_i \geq \bar{D} 
    \quad \text{ and } \quad  \hat{Y}_i \geq Y_i \quad \text{ for all } i \in I.
\end{align*}

Then we get our reference point as follows: 
\begin{itemize}
    \item The reference price for crude steel \(\bar{P}\) is the shadow price 
    of the demand constraint. 
    \item The reference capital rental rates \(\bar{R}_i\) are given 
    by the negative of the shadow prices of the capacity constraints. 
    \item Also, the production quantities of each plants that solves 
    this problem are now the reference quantities. 
    \item Then, we get the following reference quantities for factor inputs:
    \[ \bar{H}_f = \sum_{i \in I} a_{t(i), f} \bar{Y}_i \]
    \item Lastly, the reference emission level from using each factor is given by: 
    \[ \bar{e}_f = \sum_{i \in I} a_{t(i), f} \kappa_f \bar{Y}_i \]
\end{itemize}


\subsection*{Variables}
\begin{itemize}
    \item Quantity of crude steel production by plant \(\mathbf{Y}_i\)
    \item Price of crude steel \(\mathbf{P}\)
    \item Price for each factor \(\mathbf{V}_f\)
    \item Capital rental rate for each plant \(\mathbf{R}_i\)
    \item Price for carbon emission rights \(\mathbf{W}\)
\end{itemize}
\subsection*{Equations}
\begin{itemize}
    \item Zero profit condition of steel plants:
    \begin{equation}
        \sum_{f \in F} a_{t(i), f} (\mathbf{V}_f + \tau_{i, f}) + \mathbf{R}_i 
        + \sum_{f \in F} \mathbf{W} \kappa_f a_{t(i), f}
        \geq
        \mathbf{P}
        \quad\perp\quad
        \mathbf{Y}_i \geq 0
        \quad \forall i \in I
    \end{equation}
    
    \item Market clearance condition for crude steel:
    \begin{equation}
        \sum_{i \in I} \mathbf{Y}_i 
        \geq
        \bar{D} \left( 1 + \varepsilon \left(\frac{\mathbf{P}}{\bar{P}} - 1\right) \right)
        \quad\perp\quad
        \mathbf{P} \geq 0
    \end{equation}
    
    \item Market clearance condition for factors:
    \begin{equation}
        \bar{H}_f \left( 1 + \rho_f \left(\frac{\mathbf{V}_f}{\bar{V}_f} - 1\right) \right)
        \geq
        \sum_{i \in I} a_{t(i), f} \mathbf{Y}_i
        \quad\perp\quad
        \mathbf{V}_f \geq 0
        \quad \forall f \in F
    \end{equation}
    
    \item Market clearance condition for capital (plant-specific capacity):
    \begin{equation}
        \hat{Y}_i \geq \mathbf{Y}_i
        \quad\perp\quad
        \mathbf{R}_i \geq 0
        \quad \forall i \in I
    \end{equation}
    
    \item Market clearance condition for carbon emission rights:
    \begin{equation}
        \chi \left( \sum_{f \in F} \bar{e}_f \right)
        \geq
        \sum_{i \in I}\sum_{f \in F} a_{t(i), f} \kappa_f \mathbf{Y}_i
        \quad\perp\quad
        \mathbf{W} \geq 0
    \end{equation}
\end{itemize}


\end{document}