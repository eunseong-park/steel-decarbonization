\documentclass{scrartcl}
\usepackage[a4paper, margin = 3cm]{geometry}
\usepackage[english]{babel}
\usepackage{amsmath}
\usepackage{amssymb}
\usepackage{hyperref}
\usepackage{graphicx}

\setkomafont{disposition}{\normalfont\bfseries}
\newcommand{\tcooe}{\ensuremath{\mathrm{tCO_2e}}}

\begin{document}

\title{Emissions Reduction in the Global Steel Industry and Policy Measures}
\subtitle{Group Project Topic}
\author{Eunseong Park\footnote{eunseong.park@zew.de}}
\date{December 2025}
\maketitle

\section*{Context}
The steel industry is among the most carbon-intensive sectors, accounting for
approximately 7\% of global greenhouse gas emissions. Meeting the goals of the
Paris Agreement, which aim to limit global warming to well below 2°C, requires
substantial decarbonization of steel production. Unilateral climate
policies (for example, sub-national carbon taxes or emissions trading
systems) can create a risk of carbon leakage, whereby emissions shift to
countries with less stringent climate policies. This concern has prompted
calls for border carbon adjustments, which aim to level the playing field for
domestic producers and reduce incentives for leakage. Policymakers worldwide
are therefore exploring different policy instruments to lower emissions in
the steel sector.

\section*{Starting Point}
We provide a simple, single-country partial-equilibrium model of the steel
industry as a baseline.
\subsection*{Tasks}
\begin{itemize}
    \item Extend the model to a multi-region framework and simulate the
    impacts of various policy instruments (carbon taxes, emissions trading
    systems, and technology standards) on the steel sector's greenhouse gas
    emissions.
    \item Assess the effectiveness of these climate policy instruments in
    reducing emissions. Compute potential carbon-leakage rates and discuss the
    implications for policy design.
    \item Evaluate the effectiveness of border carbon adjustments in
    mitigating leakage, and analyze the trade-offs between protecting domestic
    producers and achieving global emissions reductions.
\end{itemize}

\pagebreak

\section*{A Simple Steel Industry Model}
This model is largely based on Mathiesen and Mæstad  (2004)\footnote{
    Mathiesen, Lars, and Ottar Mæstad . “Climate Policy and the Steel Industry: 
    Achieving Global Emission Reductions by an Incomplete Climate Agreement.” 
    The Energy Journal 25, no. 4 (2004): 91–114. 
    \url{https://doi.org/10.5547/ISSN0195-6574-EJ-Vol25-No4-5}.}
\subsection*{Sets}
\begin{itemize}
    \item Steel plants \(i \in I = \{1, 2, 3, ..., n\}\)
    \item Factors \(f \in F = \{\text{Iron Ore, Coking Coal, Steel Scrap, Electricity, Natural Gas}\}\)
    \item Technologies \(t \in T = \{\text{BOF, EAF, DRI}\}\)
\end{itemize}
Each steel plant will be mapped to a technology with the following notation: 
the technology of the steel plant \(i\) is \( t(i) \).

\subsection*{Parameters}
\begin{itemize}
    \item Factor use per tonne crude steel production \(a_{t,f}\):
    \begin{center}
    \begin{tabular}{ cccccc }
        \hline
        Technology & Iron Ore & Coking Coal & Steel Scrap & Electricity & Natural Gas \\
               & (tonne)  & (tonne)     & (tonne)     & (MWh)       & (tonne)     \\
        \hline
        BOF        & 1.401    & 0.653       & 0.252       & 0.033       &             \\
        EAF        &          &             & 1.026       & 0.523       &             \\
        DRI        & 0.852    &             & 0.579       & 0.707       & 7.199       \\
        \hline
       \end{tabular}
    \end{center}
    \item Reference factor market prices (USD per unit: tonne for materials, MWh for electricity) \(\bar{V}_f\):
    \begin{center}
        \begin{tabular}{ cccccc } 
            \hline
            Iron Ore & Coking Coal & Steel Scrap & Electricity & Natural Gas \\
            33.0     & 63.0        & 136.0       & 70.0        & 4.3         \\
            \hline
           \end{tabular}
    \end{center}
    \item Carbon emissions coefficients (\tcooe{} per tonne or MWh) \(\kappa_f \): 
    \begin{center}
        \begin{tabular}{ cccccc } 
            \hline
            Iron Ore & Coking Coal & Steel Scrap & Electricity & Natural Gas \\
            0.02     & 2.76        & 0.01        & 0.29        & 2.34        \\
            \hline
           \end{tabular}
    \end{center}
    \item Price elasticity of supply for each factor \(\rho_f\):
    \begin{center}
        \begin{tabular}{ cccccc } 
            \hline
            Iron Ore & Coking Coal & Steel Scrap & Electricity & Natural Gas \\
            1.0      & 2.0         & 0.5         & 0           & 0           \\
            \hline
           \end{tabular}
    \end{center}
    \item Price elasticity of demand for steel \(\varepsilon = -0.3\)
    \item Plant-specific additional costs of each factor (e.g. transport) \(\tau_{i,f}\)
    \item Plant-specific production capacity \(\hat{Y}_i\)
\end{itemize}

\pagebreak

\subsection*{Calibration and reference point}
To run the model, we set initial values for the variables. These values allow
the model to be solved, but because the model relies on simplifying
assumptions it will not reproduce every detail of empirical data. For our
analysis, we therefore construct a reference crude steel price and plant-level
production quantities that are consistent with the model's cost structure and
demand constraint; these serve as the reference point.

In the GAMS file, we create a set of fictitious plants. To make these plants
heterogeneous, we assume additive, factor-specific costs
\(\tau_{i,f} \sim U(0, 0.1)\). Capacities are drawn from
\(\hat{Y}_i \sim U(10, 15)\). For the reference production of each plant, we
assume
\(\bar{Y}_i = \hat{Y}_i - \delta_i\) with \(\delta_i \sim U(0, 5)\).
Reference demand \(\bar{D}\) is the sum of those production quantities,
\( \bar{D} = \sum_{i \in I} \bar{Y}_i\), and we solve the following problem:
\begin{align*}
    \min_{\{Y_i\}_{i\in I}} & \sum_{i \in I} \sum_{f \in F} a_{t(i), f} (\bar{V}_f + \tau_{i,f}) Y_i \\
    \text{subject to } \quad  & \sum_{i \in I} Y_i \geq \bar{D} 
    \quad \text{ and } \quad  \hat{Y}_i \geq Y_i \quad \text{ for all } i \in I.
\end{align*}

We obtain the reference point as follows:
\begin{itemize}
    \item The reference price for crude steel, \(\bar{P}\), is the shadow price
    of the demand constraint.
    \item The reference capital rental rates, \(\bar{R}_i\), equal the
    negative of the shadow prices associated with the capacity constraints.
    \item The production quantities that solve the calibration problem are the
    reference quantities for each plant.
    \item Reference factor input quantities are
    \[ \bar{H}_f = \sum_{i \in I} a_{t(i), f} \bar{Y}_i. \]
    \item Reference emissions from each factor are
    \[ \bar{e}_f = \sum_{i \in I} a_{t(i), f} \kappa_f \bar{Y}_i. \]
\end{itemize}


\subsection*{Variables}
\begin{itemize}
    \item Quantity of crude steel produced by plant, \(\mathbf{Y}_i\).
    \item Price of crude steel, \(\mathbf{P}\).
    \item Price of each factor, \(\mathbf{V}_f\).
    \item Capital rental rate for each plant, \(\mathbf{R}_i\).
    \item Price of carbon emission permits, \(\mathbf{W}\).
\end{itemize}
\subsection*{Equations}
\begin{itemize}
    \item Zero profit condition of steel plants:
    \begin{equation}
        \sum_{f \in F} a_{t(i), f} (\mathbf{V}_f + \tau_{i, f}) + \mathbf{R}_i 
        + \sum_{f \in F} \mathbf{W} \kappa_f a_{t(i), f}
        \geq
        \mathbf{P}
        \quad\perp\quad
        \mathbf{Y}_i \geq 0
        \quad \forall i \in I
    \end{equation}
    
    \item Market clearance condition for crude steel:
    \begin{equation}
        \sum_{i \in I} \mathbf{Y}_i 
        \geq
        \bar{D} \left( 1 + \varepsilon \left(\frac{\mathbf{P}}{\bar{P}} - 1\right) \right)
        \quad\perp\quad
        \mathbf{P} \geq 0
    \end{equation}
    
    \item Market clearance condition for factors:
    \begin{equation}
        \bar{H}_f \left( 1 + \rho_f \left(\frac{\mathbf{V}_f}{\bar{V}_f} - 1\right) \right)
        \geq
        \sum_{i \in I} a_{t(i), f} \mathbf{Y}_i
        \quad\perp\quad
        \mathbf{V}_f \geq 0
        \quad \forall f \in F
    \end{equation}
    
    \item Market clearance condition for capital (plant-specific capacity):
    \begin{equation}
        \hat{Y}_i \geq \mathbf{Y}_i
        \quad\perp\quad
        \mathbf{R}_i \geq 0
        \quad \forall i \in I
    \end{equation}
    
    \item Market clearance condition for carbon emission permits:
    \begin{equation}
        \chi \left( \sum_{f \in F} \bar{e}_f \right)
        \geq
        \sum_{i \in I}\sum_{f \in F} a_{t(i), f} \kappa_f \mathbf{Y}_i
        \quad\perp\quad
        \mathbf{W} \geq 0
    \end{equation}
\end{itemize}


\end{document}